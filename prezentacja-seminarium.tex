\documentclass{beamer}


\usepackage[utf8]{inputenc}
\usepackage{polski}
\usetheme{Warsaw}

\title
{System pisania testów na platformie Android}
\subtitle{Seminarium dyplomowe}
\author{Filip Malinowski}
\subject{Seminarium}


\begin{document}

  \begin{frame}
	\titlepage
  \end{frame}
  \begin{frame}
    \frametitle{Plan prezentacji}
    \begin{itemize}
    \item Opis projektu inżynierskiego
    \item Cel projektu
    \item Dostępne narzędzia
    \item Postęp prac
    \item Uzyskane wyniki
    \end{itemize}
  \end{frame}
  \begin{frame}
    \frametitle{Opis projektu inżynierskiego}
    Docelowo projekt ma zawierać:
    \begin{itemize}
    \item aplikację na systemy Android
    \item serwer do obsługi aplikacji
    \end{itemize}
    Aplikacje mają się komunikować przez Internet z serwerem oraz przesyłać do
    niego odpowiedzi studentów. Serwer ma otrzymane dane obsłużyć i przedstawić
    w formacie odpowiednim dla prowadzącego.
    
  \end{frame}
  \begin{frame}
    \frametitle{Cel projektu}
    Umożliwienie pisania testów na telefonach dając korzyści:
    \begin{itemize}
    \item prowadzącemu zajęcia
    \item studentom
    \item drzewom
    \end{itemize}
   	
  \end{frame}
  \begin{frame}
    \frametitle{Dostępne narzędzia}
    Do napisania aplikacji na telefon:
    \begin{itemize}
    \item Android studio
    \item Qt (w wersji równej 5.1 lub wyższej)
    \end{itemize}
    Do wykonania serwera:
    \begin{itemize}
    \item Nginx
    \item OpenLiteSpeed
    \item Apache
    \end{itemize}
  \end{frame}
  \begin{frame}
    \frametitle{Postęp prac}
    \begin{itemize}
    \item Androidowa aplikacja komunikująca się z serwerem Apache
    \item Skrypt do parsowania logów jako zastępstwo dla pełnoprawnego serwera
    \end{itemize}
  \end{frame}
  \begin{frame}
    \frametitle{Uzyskane wyniki}
    \begin{itemize}
    \item Pierwszy pomyślnie wykonany test
    \item Drugi test będący katastrofą
    \item Na obu testach piszących ok. 90 osób
    \item Średnio 10 zapytań na sekundę
    \end{itemize}
  \end{frame}
  \begin{frame}
  	\frametitle{Koniec}
    \begin{center}
    \Huge Dziękuję za uwagę
    \end{center}
  \end{frame}
% etc
\end{document}